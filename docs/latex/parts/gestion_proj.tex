\section{Gestion de projet}
\paragraph{} Ce projet particulièrement libre dans sa frome est également libre dans sa gestion interne.\\
Nous avons tenté d'appliquer la méthode scrum en définissant des objetifs à atteindre chaque semaine avec un courte revue en début de séace pour avoir une idée de l'avancement.\\

\paragraph{}Concernant le suivi du projet, nous avons utilisé \textbf{trello} qui permet de séparer les objectifs en t\^{a}ches et de les affecter aux membres de l'équipe.\\
Pour travailler collaborativement, nous utilisons \textbf{git} qui permet de développer les modules de façon indépendante, de les tester séparément puis de les intégrer dans une branche contenant une version fonctionnelle du projet (intégration continue).\\


\section{Bilan}
\paragraph{}\`{A} la suite de ce projet, plusieurs points importants ont été remontés par mes collègues et moi-m\^{e}me :
\begin{itemize}
\item Une \textbf{montée en compétence} sur pas mal de technologies inconnues jusqu'à lors (Python, flask). Mais nécessité de se former au début.
\item Une charge de travail un peu plus importante sur la phase finale d'intégration (malgré l'intégration continue) ainsi que pour la démonstration.
\item C'est un projet \textbf{intéressant techniquement, motivant} et très libre.
\item la liberté accordé est peut-être trop importante, un \textbf{recadrage à mi-projet} serrait judicieux afin d'estimer l'avancement du travail et éventuellement de recentrer la suite sur les fontionnalités les plus importantes/originales.
\end{itemize}