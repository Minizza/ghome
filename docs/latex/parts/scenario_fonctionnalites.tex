\section{Scénario}

Les possibilités offertes par le projet GHome sont immenses. La diversité des objets connectés qui nous sont fournis (capteurs de température, d'ouverture de porte, interrupteurs, tablettes, robot kinect) nous donne de \textbf{nombreuses }options quant à leur utilisation dans le domaine de la domotique. Mais nous ne prétendions \textbf{pas tous }les utiliser. 

Nous ne pouvions pas nous limiter, nous contraindre à intégrer tous ces éléments au détriment d'une \textbf{réelle idée}, réfléchie et implémentée en profondeur. 

C'est dans cette optique que l'équipe Minizza souhaitait avant tout adopter \textbf{un axe}, un point de vue \textbf{original }sur ce projet, qui nous enthousiasmerait et nous pousserait à aller au bout de notre projet, avant de se poser la problématique du choix de tel ou tel objet à utiliser.

\paragraph{}

C'est alors que nous est venu l'idée de la gestion d'une \textbf{arène de lasergame}. Cette idée découle du jeu \textit{Natural Selection 2}, un jeu de tir multijoueur dont la différence avec les principaux jeu du genre en fait l'attrait principal. 

Il y deux rôles essentiels dans chaque équipe strictement différents : celui de combatant, sur le terrain, \textbf{et celui de stratège}, qui a des informations supplémentaires que n'ont pas ses co-équipiers (position des ennemis repérés, direction de déplacement de ceux-ci, possibilité de construire des tours de défense, etc...). 

Il voit ce que ne peuvent voir les autres, et peut alors à lui seul établir une stratégie et \textbf{relayer l'information }en fonction de celle-ci.

Ceci nous semblait être un angle particulièrement intéressant : il s'agissait d'\textbf{augmenter les possibilités }d'un jeu extrèmement populaire de nos jours, de le transformer afin de le rendre encore plus attractif et excitant.

\paragraph{}

Nous avons alors réfléchi à la manière d'utiliser les différents capteurs pour rendre possible notre solution. De nombreuses idées ont directement découlé de notre choix de scénario en considérant les objets proposés : des \textbf{interrupteurs }pour mettre en place des plaques de pression afin détecter le passage de joueurs dans des couloirs, des \textbf{capteurs de présence }pour détecter les participants désireux de rester cachés en certains endroits de la carte, des capteurs de porte. 

Nous utiliserons aussi des \textbf{actionneurs }(essentiellement des prises de courant pilotables) à des fins diverses : des lumières pour éclairer une salle normalement sombre, des machines à fumée pour faire entrer son équipe dans une pièce surveillée par l'ennemi sans pouvoir se faire tuer.

Bien évidemment, nous comptons utiliser des \textbf{tablettes }(ou tout du moins des portables) sous Android, afin de les mettre à la disposition des chefs d'équipe. Il pourrons être informés en \textbf{temps réel }de la \textbf{position }de leurs co-équipiers et de l'\textbf{activation }des différents capteurs. Enfin, ils auront la possibilité d'\textbf{activer }tous les actuateurs mis à leur disposition directement depuis la tablette.


\section{Fonctionnalités - Cas d'utilisation}

Afin de vous exposer plus clairement toutes les manières d'aborder notre projet, nous vous proposons de parcourir l'ensemble des cas d'utilisation mis en place lors de l'élaboration de notre solution.


\subsection{Un client}
~\\
\textbf{Cas d’utilisation} : Connexion à la plateforme\\
\textbf{Acteur principal} :   tout utilisateur\\
\textbf{Post-conditions} :   l’utilisateur est connecté, les différentes pages accessibles selon ses droits\\
\textbf{Scénario principal} :
\begin{enumerate}
 \item L’utilisateur se connecte à la plateforme
 \item Il se dirige vers la page “Connection”
 \item Il rentre son identifiant
 \item Il rentre son mot de passe
 \item Il valide ses entrées
\end{enumerate}
~\\
\textbf{Extensions} :
\begin{enumerate}
 \item L’utilisateur rentre un mauvais identifiant / mot de passe
    \begin{enumerate}
     \item L’application lui signifie son erreur
     \item L’utilisateur est invité à recommencer
    \end{enumerate}
\end{enumerate}

\paragraph{}
~\\
\textbf{Cas d’utilisation} : Visualisation de l’état courant des capteurs et actionneurs\\
\textbf{Acteur principal} :   tout utilisateur\\
\textbf{Pré-conditions} : Etre connecté à la plateforme\\
\textbf{Post-conditions} :   l’utilisateur est connecté, les différentes pages accessibles selon ses droits\\
\textbf{Scénario principal} :
\begin{enumerate}
 \item L’utilisateur sélectionne la page “Devices” de la plateforme
 \item Il sélectionne un périphérique dans la liste des existants
 \item Il visualise l’état du capteur, son identifiant, sa dénomination et sa position sur la carte
\end{enumerate}


\subsection{Le gérant}
~\\
\textbf{Cas d’utilisation} : Ajout d’un périphérique\\
\textbf{Acteur principal} : le gérant\\
\textbf{Pré-conditions} : Etre connecté à la plateforme\\
\textbf{Post-conditions} : Le périphérique est ajouté en base avec toutes les informations récoltées\\
\textbf{Scénario principal} :
\begin{enumerate}
 \item L’utilisateur sélectionne la page “Devices” de la plateforme
 \item Il sélectionne le bouton “Add a device”
 \item Il choisit dans la liste déroulante le type de périphérique à ajouter
 \item Il renseigne l’identifiant “fabriquant” du périphérique (identifiant en hexadécimal)
 \item Il renseigne une dénomination pour le périphérique
 \item Il choisit son emplacement sur la carte
\end{enumerate}

\paragraph{}
~\\
\textbf{Cas d’utilisation} : Suppression d’un capteur\\
\textbf{Acteur principal} : le gérant\\
\textbf{Pré-conditions} : Etre connecté à la plateforme\\
\textbf{Post-conditions} : Le périphérique est supprimé de la base, son historique d’états également\\
\textbf{Scénario principal} :
\begin{enumerate}
 \item L’utilisateur sélectionne la page “Devices” de la plateforme
 \item Il sélectionne un périphérique dans la liste des existants
 \item Il sélectionne le bouton “Delete”
\end{enumerate}

\paragraph{}
~\\
\textbf{Cas d’utilisation} : Dessiner la carte de jeu\\
\textbf{Acteur principal} : le gérant\\
\textbf{Pré-conditions} : Etre connecté à la plateforme\\
\textbf{Post-conditions} : La carte est enregistrée en base, prête à être utilisée sur toutes les plateformes\\
\textbf{Scénario principal} :
\begin{enumerate}
 \item L’utilisateur sélectionne la page “Draw” de la plateforme
 \item Il clique sur la carte pour commencer à tracer la carte
 \item Il clique une seconde fois pour valider son trait
 \item Il recommence les deux derniers points jusqu'à complétion de la carte
 \item Il sélectionne le bouton “Export SVG” pour enregistrer la carte
\end{enumerate}

\subsection{Le chef d'équipe}
~\\
\textbf{Cas d’utilisation} : Ouvrir l’application\\
\textbf{Acteur principal} : Le gérant\\
\textbf{Pré-conditions} : Avoir été désigné les chefs d’équipe, la partie est remise à zéro pour en démarrer une nouvelle (état des capteurs, bonus à utiliser, position des équipiers,...)\\
\textbf{Post-conditions} : L’application est ouverte, la partie est lancée\\
\textbf{Scénario principal} :
\begin{enumerate}
 \item Le gérant distribue la tablette aux deux chefs d’équipe (tablettes respectivement calibrées pour connaitre allies/enemies)
 \item Le gérant demande aux chefs d’équipe de lancer l’application
 \item L’application arrive sur la fenêtre principale, les alliés sont affichés, les bonus disponibles (non grisés, sans surbrillance) et les capteurs à l’arret
\end{enumerate}

\paragraph{}
~\\
\textbf{Cas d’utilisation} :  Détecter la présence d’un ennemi\\
\textbf{Acteur principal} : Le chef d’équipe\\
\textbf{Pré-conditions} : L’application est ouverte, la partie est lancée\\
\textbf{Scénario principal} :
\begin{enumerate}
 \item Un ennemi {appuie sur un plaque de pression ; est détecté par un capteur de présence ; ouvre une porte surveillée par un capteur}
 \item Si l’application détecte le premier point, on teste si c’est effectivement un ennemi qui a déclenché le capteur
 \item Si c’est effectivement un ennemi, un son est joué et un changement du sprite du capteur en question est effectué pour signaler au chef d’équipe le déclenchement du capteur par un ennemi
 \item On affiche durant 2s l’ennemi en question, et les alentours dans un rayon de 2m
\end{enumerate}

\paragraph{}
~\\
\textbf{Cas d’utilisation} :  Sélectionner / Déclencher un actuateur\\
\textbf{Acteur principal} : Le chef d’équipe (CE)\\
\textbf{Pré-conditions} : L’application est ouverte, la partie est lancée\\
\textbf{Post-conditions} :  Une gestion post traitement est effectuée pour déclencher l’action de l’actuateur\\
\textbf{Scénario principal} :
\begin{enumerate}
 \item Le CE sélectionne l’actuateur sur la carte de jeu
 \item Des informations complémentaires à l’actuateur sont affichées (nom, type, localisation) // Type : {lumière ; machine à fumée ; porte close}
 \item Le bonus relatif au type de l’actuateur est mis en surbrillance, les autres sont ternis (seul le bonus mis en surbrillance doit être détectable)
 \item Si le CE ré-appuie sur la carte de jeu, en dehors de n’importe quel autre actuateur, celà remet l’affichage de base. S’il appuie sur un autre actuateur, celà met les informations de l’autre actuateur
 \item Lorsqu’un actuateur est sélectionné, si le CE appuie sur le bonus correspondant, il le déclenche, et le nombre de déclenchement de l’actuateur est décrémenté
 \item Lorsque le nombre d’activations d’un actuateur est tombé à zéro, le bonus lié à l’actuateur est totalement grisé, il devient inutilisable
\end{enumerate}

\paragraph{}
~\\
\textbf{Cas d’utilisation} :  Fin de manche\\
\textbf{Acteur principal} : Le chef d’équipe\\
\textbf{Pré-conditions} : Une manche a été jouée, elle vient de se terminer\\
\textbf{Post-conditions} :  Une autre manche est relancée, donnant lieu au CdU “Ouvrir l’application”, point 3\\
\textbf{Scénario principal} :
\begin{enumerate}
 \item L’équipe se retrouve autour du CE pour faire un débriefing de la manche
 \item Des informations sont affichées (Bonus utilisés par l’équipe / l’équipe adverse)
 \item La carte affiche les capteurs et leur nombre d’activations
\end{enumerate}

