\section{Demonstration pas à pas}
\subsection{Ajout d'un capteur}
\paragraph{} Il n'est possible d'ajouter un capteur que si l'on est identifié en tant qu'administrateur.

\begin{enumerate}
\item Accueil une fois connecté\\
~~\\
\includegraphics[scale=0.25]{image/homeAdmin.png}\\
\item navigation vers les périphériques\\
On peut visualiser l'ensemble des périphériques présents dans la base.
~~\\
\includegraphics[scale=0.25]{image/newPeriph.png}\\
\item choix du type de périphérique à créer\\
~~\\
\includegraphics[scale=0.25]{image/periphChoice.png}\\
\item Renseignement des champs \\
~~\\
\includegraphics[scale=0.25]{image/formPeriph.png}\\
Le Périphérique est maintenant présent dans la liste des périphériques.
\item Placement du nouveau périphérique\\
En sélectionnant le périphérique nouvellement créé on peut visualiser les informations le concernant. On peut également le placer sur la carte d'un clic.\\
\includegraphics[scale=0.25]{image/placePeriph.png}
\end{enumerate}

\subsection{Simuler le comportement des capteurs de position}
\paragraph{} Une fois identifié sur l'application, il est possible de simuler le comportement d'une capteur de position en prenant le contrôle du joueur.\\

\begin{enumerate}
 \item Aller dans la section ``Play'' de l'application.\\
 ~~\\
 \includegraphics[scale=0.22]{image/surPlay}
 \item Choisir le joueur à simuler à l'aide de la fenêtre déroulante. Le joueur choisi voit son icône modifiée.\\
 ~~\\
  \includegraphics[scale=0.22]{image/listePlay}
 \item Déplacer le joueur à l'aide des flêches.\\
\end{enumerate}

\subsection{Voir la vue ``tablette'' sur l'application}
\paragraph{} Il est possible, lorsque l'utilisateur est identifié sur l'application, d'obtenir la même vue que celle projetée sur la tablette des chefs d'équipe (à l'exception près que l'utilisateur verra les \textbf{deux} équipes). Il pourra ainsi voir en direct la détection des joueurs, mais également activer les actionneurs à sa guise.\\

\begin{enumerate}
 \item Aller dans la section ``Game'' de l'application.\\
 ~~\\
  \includegraphics[scale=0.22]{image/surGame}
 \item La vue suivante s'affiche. Si un joueur est repéré, le capteur (en jaune) correspondant clignote.\\
 ~~\\
 \includegraphics[scale=0.22]{image/captGame}
 \item Il est possible de sélectionner un actionneur (en violet). Il est alors en surbrillance, et ses informations s'affichent.\\
 ~~\\
 \includegraphics[scale=0.22]{image/infosGame}
 \item Cliquer sur le bouton ``Activer'' permet ainsi de mettre en marche l'actionneur.\\
\end{enumerate}


