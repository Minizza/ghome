\documentclass[a4paper]{report}

			%packages
\usepackage[french]{babel}
\usepackage[utf8x]{inputenc}
\usepackage[T1]{fontenc}
\usepackage{amsmath}
\usepackage{graphicx}

%%caractères spéciaux
 \usepackage{pifont}
			%config hyperref
\usepackage{hyperref}
\hypersetup{
    colorlinks,
    citecolor=blue,
    filecolor=blue,
    linkcolor=blue,
    urlcolor=blue
}
%liste de base avec des points
\renewcommand{\FrenchLabelItem}{\textbullet}
%Tableau
\usepackage{array,multirow,makecell}
\setcellgapes{1pt}
\makegapedcells
\usepackage[table]{xcolor}
\newcolumntype{R}[1]{>{\raggedleft\arraybackslash }m{#1}}
\newcolumntype{L}[1]{>{\raggedright\arraybackslash }m{#1}}
\newcolumntype{C}[1]{>{\centering\arraybackslash }m{#1}}

% Démarre une nouvelle section sur une nouvelle page
\let\stdsection\section
\renewcommand\section{\newpage\stdsection}

% Supprime les 0.x pour les sections (ne gère pas appendix)
\makeatletter 
\renewcommand\thesection{\@arabic\c@section} 
\makeatother

			%infos du docuument
\newcommand{\leTitre}{Rapport}

\title{\leTitre} 
\author{H4111}

\begin{document}

\begin{titlepage}


%----------------------------------------------------------------------------------------
%	Templates comes from: http://www.latextemplates.com/template/university-assignment-title-page
%	with some minor changes
%----------------------------------------------------------------------------------------

\newcommand{\HRule}{\rule{\linewidth}{0.5mm}} % Defines a new command for the horizontal lines, change thickness here

\center % Center everything on the page

%----------------------------------------------------------------------------------------
%	HEADING SECTIONS
%----------------------------------------------------------------------------------------

\includegraphics [width=40mm]{image/logo_INSA.png} \\[1.5cm]
%\textsc{\LARGE INSA de Lyon}\\[1.5cm] % Name of your university/college


\textsc{\large Équipe Minizza}\\[0.5cm] % Minor heading such as course title




%----------------------------------------------------------------------------------------
%	TITLE SECTION
%----------------------------------------------------------------------------------------

\HRule \\[0.4cm]
{ \huge \bfseries \leTitre}\\[0.4cm] % Title of your document
\textsc{\Large PLD GHome}\\[0.5cm] % Major heading such as course name
\HRule \\[1.5cm]
 
%----------------------------------------------------------------------------------------
%	AUTHOR SECTION
%----------------------------------------------------------------------------------------

\begin{minipage}{0.4\textwidth}
\begin{flushleft} \large
\textbf{\emph{H4111 :}}\\
Quentin \textsc{Bonnet}\\
Loric \textsc{Brevet}\\
Nicolas \textsc{Buisson}\\
Louise \textsc{Crépet}\\
Timothée \textsc{Germain}\\
Julien \textsc{Rollet}% Your name
\end{flushleft}
\end{minipage}
~
\begin{minipage}{0.4\textwidth}
\begin{flushright} \large
\textbf{\emph{Enseignants :}} \\
Marian \textsc{Scuturici}\\
Kévin \textsc{Marquet}% Supervisor's Name
\end{flushright}
\end{minipage}\\[4cm]

% If you don't want a supervisor, uncomment the two lines below and remove the section above
%\Large \emph{Author:}\\
%John \textsc{Smith}\\[3cm] % Your name

%----------------------------------------------------------------------------------------
%	DATE SECTION
%----------------------------------------------------------------------------------------

{\large Année scolaire 2013 - 2014}\\[3cm] % Date, change the \today to a set date if you want to be precise

%----------------------------------------------------------------------------------------
%	LOGO SECTION
%----------------------------------------------------------------------------------------

%\includegraphics[scale=0.2]{minizza-big.png}\\[1cm] % Include a department/university logo - this will require the graphicx package
 
%----------------------------------------------------------------------------------------

\vfill % Fill the rest of the page with whitespace

\end{titlepage}

\clearpage
\tableofcontents
\clearpage


\section{Scénario}


\section{Fonctionnalités - Cas d'utilisation}


\subsection{Un client}
~\\
\textbf{Cas d’utilisation} : Connexion à la plateforme\\
\textbf{Acteur principal} :   tout utilisateur\\
\textbf{Post-conditions} :   l’utilisateur est connecté, les différentes pages accessibles selon ses droits\\
\textbf{Scénario principal} :
\begin{enumerate}
 \item L’utilisateur se connecte à la plateforme
 \item Il se dirige vers la page “Connection”
 \item Il rentre son identifiant
 \item Il rentre son mot de passe
 \item Il valide ses entrées
\end{enumerate}
~\\
\textbf{Extensions} :
\begin{enumerate}
 \item L’utilisateur rentre un mauvais identifiant / mot de passe
    \begin{enumerate}
     \item L’application lui signifie son erreur
     \item L’utilisateur est invité à recommencer
    \end{enumerate}
\end{enumerate}

\paragraph{}
~\\
\textbf{Cas d’utilisation} : Visualisation de l’état courant des capteurs et actionneurs\\
\textbf{Acteur principal} :   tout utilisateur\\
\textbf{Pré-conditions} : Etre connecté à la plateforme\\
\textbf{Post-conditions} :   l’utilisateur est connecté, les différentes pages accessibles selon ses droits\\
\textbf{Scénario principal} :
\begin{enumerate}
 \item L’utilisateur sélectionne la page “Devices” de la plateforme
 \item Il sélectionne un périphérique dans la liste des existants
 \item Il visualise l’état du capteur, son identifiant, sa dénomination et sa position sur la carte
\end{enumerate}


\subsection{Le gérant}
~\\
\textbf{Cas d’utilisation} : Ajout d’un périphérique\\
\textbf{Acteur principal} : le gérant\\
\textbf{Pré-conditions} : Etre connecté à la plateforme\\
\textbf{Post-conditions} : Le périphérique est ajouté en base avec toutes les informations récoltées\\
\textbf{Scénario principal} :
\begin{enumerate}
 \item L’utilisateur sélectionne la page “Devices” de la plateforme
 \item Il sélectionne le bouton “Add a device”
 \item Il choisit dans la liste déroulante le type de périphérique à ajouter
 \item Il renseigne l’identifiant “fabriquant” du périphérique (identifiant en hexadécimal)
 \item Il renseigne une dénomination pour le périphérique
 \item Il choisit son emplacement sur la carte
\end{enumerate}

\paragraph{}
~\\
\textbf{Cas d’utilisation} : Suppression d’un capteur\\
\textbf{Acteur principal} : le gérant\\
\textbf{Pré-conditions} : Etre connecté à la plateforme\\
\textbf{Post-conditions} : Le périphérique est supprimé de la base, son historique d’états également\\
\textbf{Scénario principal} :
\begin{enumerate}
 \item L’utilisateur sélectionne la page “Devices” de la plateforme
 \item Il sélectionne un périphérique dans la liste des existants
 \item Il sélectionne le bouton “Delete”
\end{enumerate}


\subsection{Le chef d'équipe}
~\\
\textbf{Cas d’utilisation} : Ouvrir l’application\\
\textbf{Acteur principal} : Le gérant\\
\textbf{Pré-conditions} : Avoir été désigné les chefs d’équipe, la partie est remise à zéro pour en démarrer une nouvelle (état des capteurs, bonus à utiliser, position des équipiers,...)\\
\textbf{Post-conditions} : L’application est ouverte, la partie est lancée\\
\textbf{Scénario principal} :
\begin{enumerate}
 \item Le gérant distribue la tablette aux deux chefs d’équipe (tablettes respectivement calibrées pour connaitre allies/enemies)
 \item Le gérant demande aux chefs d’équipe de lancer l’application
 \item L’application arrive sur la fenêtre principale, les alliés sont affichés, les bonus disponibles (non grisés, sans surbrillance) et les capteurs à l’arret
\end{enumerate}

\paragraph{}
~\\
\textbf{Cas d’utilisation} :  Détecter la présence d’un ennemi\\
\textbf{Acteur principal} : Le chef d’équipe\\
\textbf{Pré-conditions} : L’application est ouverte, la partie est lancée\\
\textbf{Scénario principal} :
\begin{enumerate}
 \item Un ennemi {appuie sur un plaque de pression ; est détecté par un capteur de présence ; ouvre une porte surveillée par un capteur}
 \item Si l’application détecte le premier point, on teste si c’est effectivement un ennemi qui a déclenché le capteur
 \item Si c’est effectivement un ennemi, un son est joué et un changement du sprite du capteur en question est effectué pour signaler au chef d’équipe le déclenchement du capteur par un ennemi
 \item On affiche durant 2s l’ennemi en question, et les alentours dans un rayon de 2m
\end{enumerate}

\paragraph{}
~\\
\textbf{Cas d’utilisation} :  Sélectionner / Déclencher un actuateur\\
\textbf{Acteur principal} : Le chef d’équipe (CE)\\
\textbf{Pré-conditions} : L’application est ouverte, la partie est lancée\\
\textbf{Post-conditions} :  Une gestion post traitement est effectuée pour déclencher l’action de l’actuateur\\
\textbf{Scénario principal} :
\begin{enumerate}
 \item Le CE sélectionne l’actuateur sur la carte de jeu
 \item Des informations complémentaires à l’actuateur sont affichées (nom, type, localisation) // Type : {lumière ; machine à fumée ; porte close}
 \item Le bonus relatif au type de l’actuateur est mis en surbrillance, les autres sont ternis (seul le bonus mis en surbrillance doit être détectable)
 \item Si le CE ré-appuie sur la carte de jeu, en dehors de n’importe quel autre actuateur, celà remet l’affichage de base. S’il appuie sur un autre actuateur, celà met les informations de l’autre actuateur
 \item Lorsqu’un actuateur est sélectionné, si le CE appuie sur le bonus correspondant, il le déclenche, et le nombre de déclenchement de l’actuateur est décrémenté
 \item Lorsque le nombre d’activations d’un actuateur est tombé à zéro, le bonus lié à l’actuateur est totalement grisé, il devient inutilisable
\end{enumerate}

\paragraph{}
~\\
\textbf{Cas d’utilisation} :  Fin de manche\\
\textbf{Acteur principal} : Le chef d’équipe\\
\textbf{Pré-conditions} : Une manche a été jouée, elle vient de se terminer\\
\textbf{Post-conditions} :  Une autre manche est relancée, donnant lieu au CdU “Ouvrir l’application”, point 3\\
\textbf{Scénario principal} :
\begin{enumerate}
 \item L’équipe se retrouve autour du CE pour faire un débriefing de la manche
 \item Des informations sont affichées (Bonus utilisés par l’équipe / l’équipe adverse)
 \item La carte affiche les capteurs et leur nombre d’activations
\end{enumerate}


\section{Architecture de l'application}

Afin de rendre notre idée possible, il a fallu réfléchir à l'architecture de la solution que nous allions proposer. Pour cela, un important temps de réflexion a été entamé durant les séances d'initialisation du projet. Grâce à la définition précise des cas d'utilisation, certains choix ont été faits. Ces derniers vont porter sur les différents modules à prendre en compte ou encore le choix des technologies à utiliser. La description ci-dessous va justifier et décrire ces choix.

Tout d'abord, intéressons-nous aux différents modules utilisés ou bien développés avec la solution. La salle de laser game est le centre de la réflexion. Rappelons-nous que la position des joueurs peut être visualisée par le gérant ou encore par l'un des chefs d'équipe. Ces capteurs de position qui envoient constamment des données sont notre premier élément externe appartenant au GHome. A ajouter à ceux-ci les actionneurs permettant d'ouvrir une porte ou bien d'activer de la fumée. Ensuite, deux terminaux sont à prendre en compte : la vue du gérant accessible via un ordinateur puis la vue des chefs d'équipe qui s'interfacera à travers une tablette tactile. Afin de stocker les informations des capteurs, nous avons trouvé nécessaire d'intégrer une base de données à notre application. Pour analyser les données des capteurs, les entrer en base et permettre la configuration de notre système, un serveur sera développé. Il est le cœur de notre solution applicative.

Si l'on s'intéresse davantage à ce serveur, on peut le découper en deux parties afin de diviser les tâches et de pouvoir servir de manière asynchrone. Le traducteur sera la partie du serveur qui analysera les trames reçues des capteurs, qui les parsera et qui les entrera dans la base de données. Le traducteur aura aussi comme objectif d'envoyer les commandes des actionneurs. A savoir qu'un service de simulation appelé fake Jérôme permettra de simuler l'envoi de données des capteurs. Le traducteur se chargera donc aussi de transiter avec ce dernier.

La deuxième partie du serveur sera plus orientée utilisateur. En effet, le gérant et les chefs d'équipe ayant besoin de configurer l'application via des terminaux mobiles, une interface doit être mise en place. Pour cela nous avons décidé que toutes nos interfaces seraient orientées Web afin de centraliser le développement et de pouvoir s'ouvrir à de multiples terminaux sans efforts majeurs supplémentaires. Qui dit application Web dit serveur Web : cette deuxième partie qui s'intitulera IComm (comme "Interface de Communication") aura donc pour rôle de servir les clients web.

Bien entendu, la partie base de données est commune aux deux modules du serveur présentés ci-dessus et permettra de faire le lien entre ces deux derniers. Le scénario typique pourrait être :
- un capteur envoie une donnée
- le traducteur la reçoit, la parse et la rentre en base de données
- un client lambda requête le serveur web afin d'afficher l'état du capteur depuis sa tablette
- IComm va chercher l'information correspondante en base de données et la renvoie au client

Avec ce système tout est parfaitement divisé et les deux parties du serveur sont indépendantes l'une de l'autre. Cette division de la solution applicative nous permet aussi de diviser le travail en suivant la même logique. Une partie de l'équipe travaille sur le traducteur pendant qu'une autre travaille sur la partie interface et une dernière sera plus spécialement focalisée sur l'interaction avec la base de données.
\section{Documentation du nouveau capteur}


\section{Demonstration pas à pas}
\subsection{Ajout d'un capteur}
\paragraph{} Il n'est possible d'ajouter un capteur que si l'on est identifié en tant qu'administrateur.

\begin{enumerate}
\item Accueil une fois connecté\\
~~\\
\includegraphics[scale=0.25]{image/homeAdmin.png}\\
\item navigation vers les périphériques\\
On peut visualiser l'ensemble des périphériques présents dans la base.
~~\\
\includegraphics[scale=0.25]{image/newPeriph.png}\\
\item choix du type de périphérique à créer\\
~~\\
\includegraphics[scale=0.25]{image/periphChoice.png}\\
\item Renseignement des champs \\
~~\\
\includegraphics[scale=0.25]{image/formPeriph.png}\\
Le Périphérique est maintenant présent dans la liste des périphériques.
\item Placement du nouveau périphérique\\
En sélectionnant le périphérique nouvellement créé on peut visualiser les informations le concernant. On peut également le placer sur la carte d'un clique.\\
\includegraphics[scale=0.25]{image/placePeriph.png}
\end{enumerate}

\section{Benchmarking}
\paragraph{} La partie serveur du projet supporte assez bien la charge : la mise à jour d'une dizaine de capteur de position s'effectue avec une occupation CPU de 40\% environ.\\
La partie visualisation utilisant javascript est beaucoup plus lourde. Du fait de l'utilisation de technologie asynchrone, elle exige un polling fréquent de l'état des capteurs dans la base ce qui est très couteux.\\ 
La visualisation n'est pas fluide, le processeur est limitant (occupation proche des 100\%).\\

L'utilisation de websocket aurait permis d'améliorer grandement les performance à ce niveau au prix d'un plus gros travail.
\section{Tests pour l'application}

Les fichiers de tests sont placés dans le dossier \texttt{src/tests/}. Pour les lancer, il suffit d'utiliser la commande \texttt{python [nom du fichier]}. Les tests seront alors lancés automatiquement.

\subsection{Fichier base.py}

Il contient les tests vérifiant la bonne intégration des objets du modèle dans la base MongoDB. Note : les sorties des fichiers de test ne tiennent pas compte des messages du logger, indépendants des tests.

\begin{itemize}
	\item \texttt{user\_test()} : test l'intégration des utilisateurs dans la base.\\ 
	Sortie : \\
	\texttt{
	Gerant admin admin\\
	Joueur1 pass1 player\\
	Joueur2 pass2 player\\
	Chef1 passChef1 chief\\
	Chef2 passChef2 chief\\
	}

	\item \texttt{device\_test()} : test de l'intégration des capteurs et actionneurs dans la base MongoDB.\\
	Sortie : \\
	\texttt{
	531f1d5cc17e67140d0f2891   AB2489DE   Actionneur zombie apocalypse\\
	531f1d5cc17e67140d0f28a1   AE2489DB   Actionneur fumée bureau\\
	531f1d5cc17e67140d0f28a3   AC2489DA   Actionneur lampe cave\\
	531f1d5cc17e67140d0f28a5   AD2489DC   Actionneur porte coffre\\
	531f1d5cc17e67140d0f288f   AB4242CD   Détecteur de fin du monde\\
	531f1d5cc17e67140d0f28a7   ADEDF3E7   Equipe 1 joueur 1\\
	531f1d5cc17e67140d0f28a9   BF458ECD   Equipe 1 joueur 2\\
	531f1d5cc17e67140d0f28ab   BE458ECD   Equipe 1 joueur 6\\
	531f1d5cc17e67140d0f28ad   AB4BCFED   Equipe 2 joueur 1\\
	531f1d5cc17e67140d0f28af   AB25FECD   Equipe 2 joueur 2\\
	531f1d5cc17e67140d0f28b1   EBBD5542   Equipe 2 joueur 3\\
	531f1d5cc17e67140d0f2897   ACF24EF5   Plaque pression couloir\\
	531f1d5cc17e67140d0f2899   AD8E334D   Plaque pression hangar\\
	531f1d5cc17e67140d0f289b   A457FE6D   Détecteur présence bureau\\
	531f1d5cc17e67140d0f289d   ABFB454E   Détecteur présence salon\\
	531f1d5cc17e67140d0f289f   DF524ED5   Porte chambre\\
	531f1d5cc17e67140d0f2893   A23DF1UI09   Capteur température cuisine\\
	531f1d5cc17e67140d0f2895   A23FB45I07   Capteur température bureau\\
	}
\end{itemize}

\subsection{Fichier controller.py}

\section{Gestion de projet}
\paragraph{} Ce projet particulièrement libre dans sa frome est également libre dans sa gestion interne.\\
Nous avons tenté d'appliquer la méthode scrum en définissant des objetifs à atteindre chaque semaine avec un courte revue en début de séace pour avoir une idée de l'avancement.\\

\paragraph{}Concernant le suivi du projet, nous avons utilisé \textbf{trello} qui permet de séparer les objectifs en t\^{a}ches et de les affecter aux membres de l'équipe.\\
Pour travailler collaborativement, nous utilisons \textbf{git} qui permet de développer les modules de façon indépendante, de les tester séparément puis de les intégrer dans une branche contenant une version fonctionnelle du projet (intégration continue).\\


\section{Bilan}
\paragraph{}\`{A} la suite de ce projet, plusieurs points importants ont été remontés par mes collègues et moi-m\^{e}me :
\begin{itemize}
\item Une \textbf{montée en compétence} sur pas mal de technologies inconnues jusqu'à lors (Python, flask). Mais nécessité de se former au début.
\item Une charge de travail un peu plus importante sur la phase finale d'intégration (malgré l'intégration continue) ainsi que pour la démonstration.
\item C'est un projet \textbf{intéressant techniquement, motivant} et très libre.
\item la liberté accordé est peut-être trop importante, un \textbf{recadrage à mi-projet} serrait judicieux afin d'estimer l'avancement du travail et éventuellement de recentrer la suite sur les fontionnalités les plus importantes/originales.
\end{itemize}
\section{Vidéo de démonstration}
\url{http://www.youtube.com/watch?v=u5USqg4Es_M&feature=youtu.be}

\end{document}