\section{Documentation du nouveau capteur}

L'application GHome telle que nous l'avons concue demande l'intégration d'un nouveau type de capteur : le capteur de position, lié à chaque joueur.\\

Dans un premier temps, nous pensions intégrer cette fonctionnalité au sein d'une application que chaque joueur pourrait lancer sur un terminal mobile, utilisant alors le système GPS de ce dernier afin d'avoir la position de tout participant.\\
Nous avons très vite réalisé que ce système comporterait des défauts, qui contrediraient des besoins fonctionnels exprimés : 
\begin{itemize}
 \item 
\end{itemize}

\paragraph{}
La position de chaque joueur sera détectée par un système de triangulation mis en place dans l'arène de jeu.\\


\paragraph{}
Le capteur est utilisé grâce au format 4BS classique, dont les trames comportent les informations suivantes : 

\begin{center}
\begin{tabular}{|c|c|c|c|c|c|}
\hline
Rorg&\multicolumn{2}{|c|}{Data}&Id&Status&CheckSum\\
\hline
4242&CoordX&CoordY&Ident.Capt&FF&Calculé\\
\hline
\end{tabular}
\end{center}